\documentclass[12pt]{article}
\usepackage[top=0.0cm, left=1.3cm, bottom=1.2cm, right=1.5cm, a4paper, includehead, headsep=5mm]{geometry}
% \usepackage[T1]{fontenc}
\usepackage[utf8]{inputenc}
\usepackage[russian]{babel}
\usepackage{skak}
\usepackage{amssymb}
\usepackage{xstring}

% программа для редактирования позиций (очень удобная)
% http://alain.blaisot.free.fr/DiagTransfer/English/home.htm
% Страница для редактирования позиций на доске
% http://chess.com/analysis-board-editor

% Коды фигур
% K - король 
% Q - ферзь
% R - ладья
% B - слон
% N - конь
% P - пешка

\newcounter{boardcount}
\stepcounter{boardcount}

\newcommand{\board}[1]
{
  \StrCut{#1}{ }{\position}{\turn}
  \def\black{b}
  \begin{minipage}[t][8.3cm][t]{8.0cm}
    \fenboard{#1 - - 0 1}
    \makebox[6ex]{\raisebox{35ex}{\Large\textbf{\theboardcount\LARGE{$\blacktriangleright$}}}} \hskip 0.2cm
    \showboard
    \par
    \hskip 3.5cm
    \ifx \black\turn Ход черных \else  Ход белых \fi
  \end{minipage}
  \stepcounter{boardcount}
}

\newcommand{\pagetitle}{{\Large\bf{Защита от шаха}}\vskip 0.6cm}

\newcommand{\page}[6]
{
  \pagetitle
  \begin{tabular}{cc}
    \board{#1} & \board{#2} \\
    \board{#3} & \board{#4} \\
    \board{#5} & \board{#6} \\
  \end{tabular}
  \newpage
}

\setupboard{22pt}{12pt}

\begin{document}
\begin{center}
  % 1 - номер первой доски на листе
  \page{r1bqkb1r/pppP2pp/5n2/5pB1/2P5/2N5/PP2PPpP/R2QKB1R b}
  {2n3k1/4r3/pp2q1Rp/3Q4/3P4/8/PP2rP2/5RK1 b}
  {rnbqkbnr/pp2pppp/3p4/1Bp5/4P3/5N2/PPPP1PPP/RNBQK2R b}
  {r1b1k2r/p1p1qppp/2p2n2/3P4/1b6/2NB4/PPP2PPP/R1BQK2R w}
  {rnbqkb1r/ppp2ppp/3p1n2/8/8/3P1N2/PPP1QPPP/RNB1KB1R b}
  {r1bq1b1r/ppp2kpp/2n5/3n4/2Bp4/5Q2/PPP2PPP/RNB1K2R b}
  % 7     
  \page{r1bqk2r/pppp1ppp/2n2n2/8/1bBPP3/5N2/PPP2PPP/RNBQK2R w}
  {rnbqk1nr/pp3ppp/4p3/1BppP3/1b1P4/2N5/PPP2PPP/R1BQK1NR b}
  {rn2kb1r/1bq2pp1/p2ppn1p/1B6/3NP1P1/2N1BP2/PPPQ3P/2KR3R b}
  {rnbqkbnr/ppp2ppp/8/3P4/3p4/8/PPP1QPPP/RNB1KBNR b}
  {rnb1kbnr/ppp4p/3p1q2/4pp1Q/2BP4/2N4N/PPP2PPP/R1B1K2R b}
  {rn1qkb1r/pp2pppp/5n2/3p1b2/Q2P1B2/4P3/PP3PPP/RN2KBNR b}
\end{center}
\end{document}


% 13
\page{}
{}
{}
{}
{}
{}

% 19
\page{}
{}
{}
{}
{}
{}

% 25
\page{}
{}
{}
{}
{}
{}




